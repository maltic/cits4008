

%template setup, made it adhere to UWA standards
\documentclass[12pt, a4paper]{article}
\usepackage{graphicx}
\usepackage{amsmath}
\setlength{\oddsidemargin}{0.5cm}
\setlength{\evensidemargin}{0.5cm}
\setlength{\topmargin}{-1.6cm}
\setlength{\leftmargin}{0.5cm}
\setlength{\rightmargin}{0.5cm}
\setlength{\textheight}{24.00cm} 
\setlength{\textwidth}{15.00cm}
\parindent 0pt
\parskip 5pt
\pagestyle{plain}


%meta info
\title{CITS4008 Assignment 2}
\author{Max Ward}
%let the date get auto generated

%author list formatter. not really needed here, but always nice to have
\newcommand{\namelistlabel}[1]{\mbox{#1}\hfil}
\newenvironment{namelist}[1]{%1
\begin{list}{}
    {
        \let\makelabel\namelistlabel
        \settowidth{\labelwidth}{#1}
        \setlength{\leftmargin}{1.1\labelwidth}
    }
  }{%1
\end{list}}



%-------------------------------------------------------------------------

\begin{document}


%title first!
\maketitle

%was not sure if abstract was required
%but it never hurts to have a quick skim reader friendly summary
\begin{abstract}
RNA is the messenger molecule for DNA. It also plays a vital role in cellular metabolism. As a result, it is valuable to predict the secondary structure of RNA molecules. I have found that the RNA folding problem is equivalent to finding the maximum-weight independent set of a circle graph. In addition the Nussinov algorithm, which is used to fold RNA, can be used to solve the maximum-weight independent set problem, and is competitive with similar algorithms.
\end{abstract}

\tableofcontents
\clearpage

%assignment is too small to require numbered sections, or chapters
\section*{Introduction} 
Ribonucleic Acid (RNA) is a biologically active molecule similar in structure to DNA. Recent research has found myriad functions for RNA. For example, RNA can act as a catalyst for mRNA splicing, peptide bond formation, and can alter the regulation of genes
\cite{xu2012statistical}. Because of its inherently single stranded nature, RNA forms bonds with itself, folding into
secondary and tertiary structures \cite{conn1998rna}.

It is axiomatic that chemical structure is tantamount to biological function, and RNA is no exception. For this reason, there has and continues to be an intense
interest in predicting the secondary structure of RNA
molecules in silico. This is  because it will allow the detection and classification of unknown RNAs, and assist the design of new RNA based drugs \cite{condon2003problems}. The secondary structure of RNA
is also highly conserved during evolution, indicating its importance \cite{hofacker2008rna}.

I endeavour to give the reader a brisk but incisive review of RNA secondary structure prediction algorithms in this paper. For the sake of succinctness, my focus shall be pseudoknot-free RNA prediction methods. Particularly those able to predict RNA structures ex nihilo---that is with no information other than the primary RNA sequence itself.

\section{Dynamic Programming}
\subsection{The Nussinov Algorithm}
According to the thermodynamic hypothesis, biologically active molecules should form structures that have minimal free energy and thus maximum stability [ref]. For RNA molecules, maximising bonds is a crude but nonetheless accurate measure of energetic stability, as every bond increases the
stability of a structure \cite{nussinov1978algorithms}. In the late 1970s, when the first large RNA molecules
were being successfully sequenced, Nussinov et al. \cite{nussinov1978algorithms} introduced an algorithm to find a single structure with the maximal number of bonds using dynamic programming. This is possible only with the restriction that all bonding pairs are nested, and hence contains no pseudoknots.

\begin{figure}
\begin{center}
%the image is quite large, needs to be scaled dwon
\scalebox{0.3}{\includegraphics{figure1}}
\end{center}
\caption{RNA secondary structure as described in the Nussinov algorithm.
Taken from a publication by Nussinov \& Jacobson \cite{nussinov1980fast}.}
\label{nuss_rna}
\end{figure}


Because of its dynamic programming nature, this algorithm performs recursive decompositions of an RNA and builds
larger structures out of repeated substructures. A natural representation of this is
depicted in Figure \ref{nuss_rna}. Part A of Figure \ref{nuss_rna} shows bonds as arcs across a circular
graph. Once can also see the nested nature of the structures being explored by the
Nussinov algorithm. Part B shows how these structures translate to actual RNAs,
and how these appear in vivo.

%you cant have several aligned equations in the same block
%unless you install a special package
%I have faked it here

\begin{equation} \label{eq:nuss_eq}
	M(i, j) = \max \left\lbrace A, B, C, D \right\rbrace
\end{equation}
\[
A = M(i, j-1)
\]
\[
B = M(i+1, j)
\]
\[
C = M(i+1, j-1) + W(i, j)
\]
\[
D = \max \left\lbrace M(i, k) + M(k+1, j) \right\rbrace \: when \: i < k < j
\]


Equation \ref{eq:nuss_eq} describes the recurrence relation for the Nussinov algorithm. The first two cases ($A$ and $B$) find the score associated with not allowing the positions $i$ and $j$ to bond. The case $C$ conversely determines the score given that positions $i$ and $j$ are bonded. The final case $D$ computes the score associated with a bifurcation. A bifurcation here means decomposition of the RNA into two separate structures.


\subsection{The Zuker Algorithm}
Soon after the work of Nussinov \& Jacobson, Zuker \& Stiegler \cite{zuker1981optimal}
described an altered version of the same algorithm which, instead of maximising
base pair weights, minimized free energy. This was done
by introducing a number of thermodynamic rules for canonical structures like
hairpin loops, internal bulges, multiloops, unbonded base pairs, and stacked base
pairs. The algorithm is similar to the Nussinov algorithm
but requires another mutually recursive dynamic programming recurrence to inject
a complex and relatively comprehensive thermodynamic scoring system.


\begin{figure}
\begin{center}
\scalebox{0.27}{\includegraphics{figure4}}
\end{center}
\caption{Diagram of faces used in the Zuker algorithm. Taken from original
publication \cite{zuker1981optimal}.}
\label{zuk_struct}
\end{figure}

First I shall introduce some useful terminology which should clarify aspects of Zuker \& Stiegler’s
algorithm. The bases of an RNA molecule can be thought of as vertices in a planar graph. Edges between such vertices are then represented as chords on a semicircular diagram (see Figures \ref{nuss_rna} and \ref{zuk_struct}). These chords are not allowed to touch. A chord
is admissible if it represents a chemically valid bond, and an
admissible structure is a structure whose graph contains only admissible bonds.
Thence, one can define a face of such a graph as any planar region bounded on all
sides---either by chords, or the edge of the graph. The
folding algorithm of Zuker \& Stiegler considers such faces as the basic contributing factor to a molecule's stability, unlike the algorithm of Nussinov \& Jacobson
which considers only individual bonds.


Let $E(F)$ represent the energy of a face $F$; inadmissible structures are given
an energy value of infinity. In addition let $V(i, j)$ be defined as the minimum free
energy of all structures in which bases $i$ and $j$ are bonded, and let $W(i, j)$ represent
the minimum free energy of all structures contained within bases $i$ and $j$ inclusive.
Note that for $W(i, j)$ the bases at $i$ and $j$ need not be bonded. Alternatively if $i$ and $j$ cannot bond, then $V(i, j) = \infty $. Finally note that $FH(i, j)$ represents a
hairpin loop from $i$ to $j$, and that $FL(i, j, i' , j' )$ is defined as the region
bounded by the bonds $i, j$ and $i', j'$. Examples of these decompositions are shown
diagrammatically in the right half of Figure \ref{zuk_struct}. In the accompanying left
half the same RNA structure is shown as it would appear in vivo.

\begin{equation} \label{eq:zuk_v1_eq}
V(i, j) = \min \left\lbrace E1, E2, E3 \right\rbrace
\end{equation}
$$E1 = E(FH(i, j))$$
$$E2 = \min \left\lbrace E(FL(i, j, i', j')) + V (i', j') \right\rbrace \: where \: i < i' < j' < j$$
$$E3 = \min \left\lbrace W (i + 1, i') + W (i' + 1, j - 1) \right\rbrace \: where \: i + 1 < i' < j - 2$$


As defined in Equation \ref{eq:zuk_v1_eq}, $V (i, j)$ is computed by minimizing
three cases. The first case considers the bond between $i$ and $j$ closing off a hairpin
loop (H in Figure \ref{zuk_struct}). The second accounts for situations in which $i$ and $j$ are bonded. This can result in a bulge (BU in Figure \ref{zuk_struct}), internal loop (I in Figure \ref{zuk_struct}), or the continuation of a stacking region with the
interior bond $i',j'$ (S in Figure \ref{zuk_struct}). The third and final case considers bifurcations
(BF in Figure \ref{zuk_struct}).

\begin{equation} \label{eq:zuk_v1_eq2}
W (i, j) = \min \left\lbrace W(i + 1, j), W(i, j - 1), V(i, j), E4 \right\rbrace
\end{equation}
$$
E4 = \min \left\lbrace W (i, i') + W (i' + 1, j) \right\rbrace \: where \: i < i' < j - 1
$$

Equation \ref{eq:zuk_v1_eq2} is the recurrence for $W(i, j)$ as described by Zuker \& Stiegler.
Again there are three cases. The first two cases $W (i + 1, j)$ and $W(i, j - 1)$ are conceptually a single case in which there is no bond between $i$ and $j$. This is similar to cases $A$ and $B$ from the Nussinov algorithm (Equation \ref{eq:nuss_eq}). The second case considers taking the bond from
$i$ to $j$. The final case allows for bifurcations in which two bonding pairs split
the structure into two sections. The minimum free energy of the best structure is defined by $W(1, n)$, where $n$ is the length of the RNA molecule. It should
be noted that the free energy for small molecules (fewer than 6 nucleotides in length) can easily be
precomputed, and forms the base case of the given recurrence relations. Because of
its efficiency ($O(N^3)$ time and $O(N^2)$ space), robustness, and extensibility, this method is,
even today, still the most popular available. The most widely used packages for RNA secondary structure prediction all contain implementations of the Zuker algorithm \cite{lorenz2011viennarna, reuter2010rnastructure}.


\subsubsection{Thermodynamic Model Improvements}

\section{Stochastic Context Free Grammars}
\section{Unification and Comparison}

\section{Conclusion}






%let bibtex do all the hard work
\bibliographystyle{plain}
\bibliography{assignment_two}


\end{document}

